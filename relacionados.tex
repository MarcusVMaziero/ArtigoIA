Em \cite{piotto:2016} é apresentada uma abordagem para extração de características de imagens visando o melhoramento da acurácia para o reconhecimento facial. Esta abordagem é baseada na teoria de redes complexas a qual fornece as medidas para remoção de conexões menos significativas para o processo de aprendizado de máquina. Com esse procedimento foi possível obter acurácia de 95\% e 96.3\% nas bases \textit{Caltech Face Dataset} (CFD)  e \textit{Head Pose Image base} (HPI).

Em \cite{sahla:2016} criam um sistema para analizar emoções em sala de aula. Aplicam redes neurais convolucionais (\textit{Convolutional Neural Network} - CNN) para classificar as expressões faciais dos alunos e professor. A aula é gravada em vídeo, na sequência esse é convertido em \textit{frames}; utilizam o algoritmo \textit{Viola-Jones} para detecção facial, ao resultado é aplicado o descritor de texturas \textit{Local Binary Pattern} (LBP), por fim, é utilizado CNN para classificação das emoções. O sistema foi testado com 105 imagens retiradas das bases CK, JAFFE e \textit{google images} as quais incluem fotos de pessoas sozinhas ou em grupos. As expressões foram classificadas como raiva, nojo, felicidade, neutralidade, supresa, medo e tristeza. Obteve, respectivamente percentual de acerto de: 33.3\%, 46,6\%, 60\%, 46,6\%, 0\%, 33,3\%, 53,3\%. No teste com a gravação em vídeo o sistema acertou 80,9\% em média e erro de 19,1\%. 

Em \cite{cruz:2019} apresenta uma pesquisa de reconhecimento de emoções humanas por imagens faciais, com a abordagem \textit{Single Shot Facial Expression Recognition} (SSFER), demonstra que o método MMOD-CNN foi o que apresentou melhor acurácia 91.89\%, para comprovar é apresentado o resultado de experimentos combinando CNN e classificador, com cinco CNN e quatro classificadores, são eles VGGNet, InceptionResNetV2, InceptionV3, MobileNetV2, ResidualNet, Softmax, SVM, Random Forest e KNN respectivamente, a pesquisa também demonstra um experimento real com vinte e sete estudantes do ensino médio utilizando do MMOD-CNN para reconhecer as emoções dos alunos.