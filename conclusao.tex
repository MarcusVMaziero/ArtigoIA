Neste trabalho foi realizado o reconhecimento de expressões faciais com aprendizado de máquina. Foram comparadas com base na acurácia o melhor desempenho para os descritores LBP e Gabor. O primeiro foi que obteve melhor resultado. Foram comparadas diferentes técnicas de normalização sendo que os testes realizados com o descritor Gabor foi o que apresentou maior sensibilidade para este tipo de alteração. Além disto foi aplicada a técnica de redução de dimensionalidade com o PCA, novamente os resultados baseados no descritor Gabor foi o mais afetado.

Por fim, coletou-se os indícios de qual a combinação de descritor, técnica de escalonamento, redução de dimensionalidade realizamos a validação cruzadada estratificada para obter estimativas mais precisas. Com isto ocorreu notável queda no desempenho dos classificadores.

Como trabalhos futuros pode-se buscar novas bases de imagens. Identificar regiões das faces específicas e extrair as características manualmente com os filtros utilizados para ser possível a regulação dos parâmetros do filtro Gabor e LBP. Com relação aos classificadores utilizar Grid Search e validação cruzada para selecionar os parâmetros exigidos pelos algorítmos de classificação utilizados. Pode-se também utilizar algorítmos de \textit{Deep learning}.
