O reconhecimento de expressões faciais são um campo especifico e instigante da área de Inteligencia Artificial (IA), conforme descreve \cite{murtaza:2019}, pois possibilita encontrar e rastrear movimentos incomuns na face das pessoas.

Com o rastreamento desses dados é possível por intermédio classificar as mesmas e compreender como as emoções afetam as tomadas de decisões, conforme \cite{hieida:2018} é um ponto instigante a ser aprofundado.
Em uma perspectiva de utilizar e melhorar o desempenho das aplicações que reconhecem e classificam expressões faciais é utilizado já por algumas empresas o Machine Learning (ML) termo em inglês que remete ao aprendizado de maquina, conforme apresenta \cite{ray:2019}, a utilização do ML já uma prática familiar em algumas empresas.

Sendo assim a presente pesquisa foi desenvolvida com o intuito de auxiliar os campos pesquisados e apresentar resultados de práticas e técnicas de reconhecimento facial.