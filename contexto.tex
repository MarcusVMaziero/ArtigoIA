A classificação de emoções por reconhecimento facial é algo que pode ser utilizado para favorecer a área escolar, conforme \cite{sahla:2016}, as emoções podem descrever a maturidade de aprendizado do discente e como o mesmo está diante dos ensinos propostos. Com isso essa classificação auxilia o docente para aumentar o desempenho da sua turma de alunos, melhorando assim o ensino.

Esse tipo de classificação também pode ser usado para fins sociais e comerciais para aumentar desempenhos de vendas e melhorar a abordagem das empresas relacionadas as frentes descritas como apresenta \cite{ozcan:2019} em sua pesquisa. Dessa forma é possível dizer que a classificações de expressões faciais podem auxiliar diversas áreas e que com o avanço desta resulta em diversas melhorias para a educação, comércio, medicina e atividades sociais.